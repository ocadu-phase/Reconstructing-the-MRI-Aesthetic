\subsection{Generating Geometry}

\subsubsection{Marching Cube Algorithm}
Given a set of slices of a 3D image we can identify the vertices and edges by looking at the edges of the masks for each segmentation.  Once this is accomplished a marching cube algorithm can be used to create a 3D surface by creating triangular connections between the vertices. This is done by performing an exhaustive iterative search throughout the entire image plane\cite{lorensen1987marching}.\\  

Intuitively, the algorithm fists divides the 3D region into cubes.  For each cube you compute weather an edge or a vertex lies within the cube and where it passes through the edge of the cube. If these conditions are met cube is part of the geometry.  After this is done you triangulate the surface to reduce redundancies and also to create a mesh. This is done by identifying where the object passes through the edges of the cube.  To create the isosurface you connect these junctions together and remove redundant connections.\\

Smoothing can also take part here but Trevor suggested not doing so because in the medical field altering the data is not acceptable.\\

\subsubsection{Marcus Method}

\subsubsection{Jawa Method}